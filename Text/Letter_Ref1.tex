\documentclass[12pt]{article}

\usepackage[margin=1in]{geometry}
\usepackage[utf8]{inputenc}
\usepackage{multirow}
\usepackage{booktabs}
\usepackage{amsmath}
\usepackage{verbatim}
\usepackage{setspace}
\onehalfspacing
\usepackage{graphicx}
\usepackage{tabularx}
\usepackage{indentfirst}
%\usepackage{subfigure}
\usepackage[titletoc,title]{appendix}
%\usepackage{apacite}
\usepackage{changes}
\usepackage{natbib}
\setcitestyle{aysep={}}					%remove comma between author and year when using citep{}.
\usepackage{color}
\usepackage{longtable}
\usepackage{lscape}
\usepackage{pdflscape}
\usepackage{caption}
\usepackage{subcaption}
\graphicspath{{Outputs/Figures/}}
\newcommand{\annote}[1]{\parbox{\textwidth}{\renewcommand{\baselinestretch}{1.0}\vspace{12pt} \footnotesize Notes: #1}}
\renewcommand{\vec}[1]{\mathbf{#1}}
\usepackage{libertine}
\usepackage{hyperref}
\hypersetup{
     colorlinks   = true,
     urlcolor	  = blue,
     citecolor    = black,
     linkcolor 	  = black
}

\begin{document}


\noindent Revision of R.D. Gabriel, `Monetary Policy and the Wage Inflation-Unemployment Tradeoff' \\
\noindent European Economic Review, Ms. Ref. No.: EEREV-D-22-00516 \bigskip

\begin{center}
\textbf{Reply to Referee 1}
\end{center}

\bigskip

\noindent Dear Referee:

\noindent I would like to kindly thank you for the detailed comments and suggestions in your review of our paper\ submitted to the \textit{European Economic Review}. The Editors have given me the opportunity to resubmit a revised version of the paper. I have prepared a revision of the manuscript paying particular attention to the comments made by you and one other Referee. In this reply, I explain how I have addressed your comments and suggestions in the revised version. I follow the argument in your report and transcribe your comments for your convenience.

\bigskip

\paragraph{Main Comments}

\begin{enumerate}
\item \textbf{I don't understand exactly where the identification comes from. You state (p10) that the IV can only be computed when a country's exchange rate is fixed with respect to a baseline country. Does it mean your estimates only pertain to these countries? This would also mean that the variation that you exploit changes as more or fewer countries join the pool with a fixed exchange rate? This could matter for your results over different samples or conditioning on the average inflation rate. It would be helpful to report (like table 1), the list of countries that you use t different points in time. (Relatedly, I didn't find table 1 very useful, but it could be more useful by displaying the information you exploit for identification).}

Yes, the IV estimates in the paper only pertain to countries in a fixed exchange rate regime. In my sample, the baseline countries are usually the U.S. or Germany. The reason for the former is that the proposed instrument only works when this is the case. I have now revised the exposition of the instrument’s construction in equation 2 (page ??) to make it clearer.

I understand the concern that the variation changes as more or less countries join the pool with a fixed exchange rate, and this might matter for my results over different subsamples. In order to address this, I start by following the first suggestion and present a new figure where I explicitly show which countries are being used at which point in time when estimating the Panel LP-IV from Equations 4 and 5. Then, I perform a robustness check where I keep the sample composition fixed using only the same set of countries (A,B,C,??) and re-do the main results. These robustness exercises can be found in Figures ??? – and show that ??

I feel that the goal of Table 1 is to present the characteristics of the newly assembled data and hint to the level and stability of inflation measures in different historical periods that is then explored in Section 4 with a more rigorous econometric approach. Nevertheless, I am sympathetic with the comment in brackets and if both referees agree that the value added of Table 1 is minimal, in a further revision, I would be comfortable to put the Table in appendix.


\item \textbf{Regarding the state-dependent effects, how can you tell that the results for state dependence in inflation are not driven by different sample periods? Different sample periods could capture different regimes (hence different Phillips multipliers) and also different average inflation. More generally, I would be a bit more careful. You are only documenting a correlation between the pi-ur trade-off and the average level of inflation. Relatedly, I would get rid of the last sentence in the conclusion. We don't know if this is true.} 

Talk about regime fixed-effects exercise plus zoom in one sub-sample using higher frequency data (quarter).

I also revised my conclusion accordingly.

\end{enumerate}


\paragraph{Other comments}

\begin{enumerate}
\item \textbf{Why do you leave out data from the inter-war period? How does that affect the result? What is the rational for removing it?}

a.	Merge with question 4 O.C.?
b.	In the previous version, I was only leaving the inter-war period out in one exercise (corresponding to Figure 4). The reason is: ???. Figure was moved to Appendix and all exercises in the main text include Inter-war period - refer to Heat Map of the sample.

\item \textbf{I'm not sure I understand the lines in Figures 4 and 5. Is the plain line the linear estimate? Does the error-band correspond to uncertainty around the linear estimate? Is it the weak-IV robust error-band?}



\item \textbf{More importantly, we would need the error-bands for the state-dependent estimates. Preferably weak-IV robust. And then a formal test that the estimates are statistically significantly different at some horizon. It's fine if it's not significant, but this should be reported.}

The requested results were reported in Appendix: Tables A.6 and A.7. I acknowledge that the exposition about this was poor, so I heavily rewrote the section to make it clearer that a formal test is being performed. I am now also including the error bands for the state-dependent estimates and removed it from the baseline estimate which is now the dotted line, nevertheless I should emphasize that these confidence bands can't be used to compare the estimates and a specific test must be perform \citep{Ramey2018} which I am doing and presenting in Tables A.6 and A.7.

\item \textbf{Why is there a huge price puzzle in Fig 4 but not in Fig3?}

Because these Figures are using different samples. While Figure 3 and 5 use the fully available sample for countries in a fixed exchange rate – now described in Table ??. Figure 4 contrasts the years when the price level was targeted - the last 20 years (2000-2020) and the Gold Standard epoch (1870-1913) - and compare them against the post-war period (1946-1999) leaving the between-war period (1920-1938) out of this analysis. Link to question 1OC.

\item \textbf{Maybe you want to normalize the IR of unemployment to peak at one, instead of the IR of the policy rate. This could be clearer to the reader so your focus is on the ratio of the IRs of pi and ur anyway.}

The normalization would not change the cumulative IRFs per se and thus, I still prefer having the interpretation of what happens to unemployment/inflation when there is an increase in the interest rate by 100 basis points?

\end{enumerate}

\paragraph{Expositional comments}

\begin{enumerate}
\item \textbf{The introduction could be better focused: start directly with what you do: the method LP to get the Phillips multiplier using IV early on and describe the IV and the intuition for the identification. (right now it comes too late and it's not explained. The identification is the core of the paper.}

I rewrote the introduction and tried to emphasize the method and the identification strategy used already in the second paragraph, with a focus on intuition where I explore the trilemma of international finance.

\item \textbf{The data contribution is minor, just compiling two datasets built by others. No need to overemphasize it.}

I rewrote the data contribution part and revised the online data appendix with all the different data sources used in the data compilation.

\item \textbf{The paper states "this is the first paper to bring a historical perspective on the … inflation—unemployment trade-off". This is a bit exaggerated, the first paper on the topic by Phillips precisely used historical data…}

I am sympathetic to this comment. I thus rewrote my contribution to make it seem less presumptuous and squared it with literature addressing the topic in a similar fashion. For example, you can now read “” on page ??

\item \textbf{Figure 2. I don't understand the series for inflation. Why is there no big increase in inflation in the 70s, early 80s?}

There are two explanations for that. First, Figure 2 plots an average across 18 countries that are equally being equally weighted so for example, the high inflation felt in the U.S. is being smoothed by the relatively lower price inflation felt by Switzerland during the same period. Secondly, it is a 20-year rolling window of cpi inflation meaning that the high inflation that some countries felt in the 70s and early 80s are being countered by the relatively lower and less volatile inflation felt in the 50s and 60s.
To show that the correlation is not being driven by this equal weights average, I run an additional exercise where I plot the same cpi inflation 20’year moving average weighting countries according to their GDP size. Results are displayed in Figure ??

Add scatter plot for each country.

\end{enumerate}

\paragraph{Minor comments}

\begin{enumerate}
\item \textbf{The author uses a number of adjectives that are either not precisely defined, or not needed and distracting. Eg., p1 "carefully" identified shocks, "clean" identification strategy}

\item \textbf{P12, "central banks have sufficient ability". Don't say sufficient. We don't know what sufficient would be here. Just say "substantial"}

\item \textbf{On p3, I suggest writing "according to a standard instead of "according to the NK model".}

I implemented all suggestions from minor comments including a, b, and c.

Regarding the use of a standard New Keynesian Model, I included in appendix a derivation for a panel framework on a microfounded New Keynesian wage Phillips curve following very closely the work from Galí (2011) and Erceg et al. (2000) on staggered wage contracts. I also include a calibration exercise emphasizing the relation between the wage stickiness parameter and the slope of the New Keynesian wage Phillips curve.

I understand that the contribution from this exercise is minimal but it is one way to give more structure to my hypothesis and explain what I actually mean when referring to a “standard New Keynesian model”. Nevertheless, if the referees think that this is unnecessary, I can remove this section from the appendix.
\end{enumerate}

Finally, in an effort to be the change I want to see in the profession, I want to state that I created an open repository in GitHub with the entire workflow from this project including all the codes, data, outputs, documentation, and tex files. This is available to everyone. So, feel free to check it out. The GitHub repository can be found on this \href{https://github.com/RicardoGabriel/Monetary-Policy-and-the-Wage-Inflation-Unemployment-Tradeoff}{GitHub repository}.

\renewcommand\bibname{References}

\begin{singlespace}
	\setlength{\bibsep}{5pt}
	\bibliography{Bibliography}
	\bibliographystyle{chicago}
\end{singlespace}



\end{document}
